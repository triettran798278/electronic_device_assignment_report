\section{Answer to Questions}
\subsection{Research on Current Sensors}
Research on the Internet and list 5 different current sensors that you can find. Along with each current sensor, please: 

(1) Give a reference source. 

(2) Give the maximum current that the sensor can measure. 

(3) Explain how to obtain its values (e.g, using ADC, UART, I2C or SPI and so on).

\begin{table}[H]
\centering
\renewcommand{\arraystretch}{1.2}
\setlength{\tabcolsep}{6pt}

\begin{tabularx}{\linewidth}{|
>{\raggedright\arraybackslash}p{2.8cm}|
>{\raggedright\arraybackslash}p{3.2cm}|
>{\raggedright\arraybackslash}p{3.6cm}|
>{\raggedright\arraybackslash}X|}
\hline
\textbf{Current Sensor} & \textbf{Max Current} & \textbf{Interface} & \textbf{Source} \\
\hline

WCM3720A & 20\,A & Digital (UART) &
\url{https://evelta.com/wcm3720a-12v-20a-ac-current-sensor-module-digital-data-output-uart/} \\
\hline

ACS758 &
$\pm 50$\,A, $\pm 100$\,A, $\pm 150$\,A, $\pm 200$\,A (IC-dependent) &
Analog &
\url{https://www.alldatasheet.vn/datasheet-pdf/pdf/533456/ALLEGRO/ACS758_13.html} \\
\hline

INA219 &
Up to $\approx 3.2$\,A (module-dependent) &
I$^2$C &
\url{https://www.ti.com/product/INA219} \\
\hline

ACS712 &
$\approx \pm 8$\,A (some modules claim $\pm 20$\,A) &
Analog (ADC) &
\url{https://www.allegromicro.com/-/media/files/datasheets/acs712-datasheet.ashx} \\
\hline

INA226 &
30\,A (module-dependent) &
I$^2$C &
\url{https://www.ti.com/lit/ds/symlink/ina226.pdf} \\
\hline
\end{tabularx}
\end{table}

\subsection{SW1 Voltage in ON and OFF States}
When SW1 OFF: $V_{SW1} = 3.3V$ because there is only a single current path.

When SW1 ON: $V_{SW1} = 0V$ because current preferentially flows through paths with lower resistance, while the path from the $3.3\,\text{V}$ supply has a relatively high resistance of $10\,\text{k}\Omega$.

\subsection{Voltage of ADC1\_CH7 and ADC1\_CH6}
\subsection*{1. Voltage calculation at ADC1\_CH7 (Reference voltage)}

The circuit uses op-amp U3B configured as a voltage follower (buffer).
The non-inverting input (pin 5, $+$) is driven by a voltage divider formed by
$R_5 = 1\,\text{k}\Omega$ and $R_8 = 1\,\text{k}\Omega$.

\[
V_{\text{ADC1\_CH7}} = V_{CC} \times \frac{R_8}{R_5 + R_8}
= 3.3\,\text{V} \times \frac{1}{1 + 1}
= 1.65\,\text{V}
\]

\subsection*{2. Voltage calculation at ADC1\_CH6 (Measured signal)}

The input signal $\text{ADC\_IN}$ passes through a voltage divider consisting of
$R_{11} = 1\,\text{k}\Omega$ and $R_{12} = 1\,\text{k}\Omega$ before entering the
non-inverting input of op-amp U3A.
Op-amp U3A is configured as a non-inverting amplifier.

\begin{itemize}
  \item Voltage at pin 3 (non-inverting input):
  \[
  V_{+} = V_{\text{ADC\_IN}} \times \frac{R_{14}}{R_{11} + R_{14}}
  = 1.65\,\text{V} \times \frac{1}{1 + 1}
  = 0.825\,\text{V}
  \]

  \item Amplifier gain:
  \[
  \text{Gain} = 1 + \frac{R_{16}}{R_{15}} = 1 + \frac{1}{1} = 2
  \]

  \item Output voltage:
  \[
  V_{\text{ADC1\_CH6}} = V_{+} \times \text{Gain}
  = 0.825\,\text{V} \times 2
  = 1.65\,\text{V}
  \]
\end{itemize}

\subsection{Low-Pass Filter Design}

Consider the signal at pin 3 of the LM358 op-amp (U3A) as shown in Figure~1.5.

The equivalent circuit is:
\[
\text{ADC\_IN} \rightarrow R_{11} \,(1\,\text{k}\Omega) \rightarrow (R_{14} \parallel C_{12}) \rightarrow \text{GND}
\]

The cutoff frequency of the low-pass filter is given by:
\[
f_c = \frac{1}{2\pi} \times \frac{R_{11} + R_{14}}{R_{11} \times R_{14} \times C_{12}}
\]

Substituting the initial values:
\[
f_c =
\frac{1}{2\pi}
\times
\frac{1000 + 1000}{1000 \times 1000 \times 100 \times 10^{-12}}
\approx 3.18\,\text{MHz}
\]

Since the desired cutoff frequency is $f_c \approx 10\,\text{kHz}$, three approaches can be considered:

\subsubsection*{Method 1: Reduce $C_{12}$ while keeping $R$ constant}

\[
C_{12} =
\frac{1}{2\pi}
\times
\frac{R_{11} + R_{14}}{R_{11} \times R_{14} \times f_c}
=
\frac{1}{2\pi}
\times
\frac{1000 + 1000}{1000 \times 1000 \times 10 \times 10^{3}}
\approx 3.18\,\text{nF}
\]

\subsubsection*{Method 2: Increase $R$ while keeping $C_{12}$ constant}

\[
R =
\frac{R_{11} \times R_{14}}{R_{11} + R_{14}}
=
\frac{1}{2\pi \times f_c \times C_{12}}
=
\frac{1}{2\pi \times 10 \times 10^{3} \times 100 \times 10^{-12}}
\approx 159\,\text{k}\Omega
\]

\subsubsection*{Method 3: Compromise between $R$ and $C$}

\[
R \times C_{12} =
\frac{1}{2\pi \times f_c}
=
\frac{1}{2\pi \times 10 \times 10^{3}}
\approx 1.59 \times 10^{-5}
\]

This approach allows proper scaling while avoiding excessively large resistor values
or impractically small capacitor values.

\subsection{LED Control}

The LED is modeled using a diode in the simulation.
The standard diode in PSpice has a forward voltage $V_f \approx 0.65\,\text{V}$.
To emulate an LED with $V_f \approx 2\,\text{V}$, a DC voltage source of $1.45\,\text{V}$ is added in series
for simulation convenience (not recommended in practical implementation).

The current through each LED is calculated as:
\[
I = \frac{V_{CC} - V_f - V_{CE(\text{sat})}}{R}
= \frac{3.3 - 2 - 0.2}{330}
\approx 3.33\,\text{mA}
\]

If a $100\,\text{mW}$ LED is required, the following approaches can be considered:

\begin{enumerate}
  \item Use a constant-current driver.
  \item Select a suitable transistor or MOSFET.
  \item Adjust the circuit parameters as follows:
\end{enumerate}

We have:
\[
I = \frac{V_{CC} - V_f - V_{CE(\text{sat})}}{R}
= \frac{V_{CC} - 2 - 0.2}{R}
= \frac{100}{0.7}\,\text{mA}
\]

Thus:
\[
\frac{V_{CC} - 2 - 0.2}{R} = \frac{100}{0.7}
\quad (\text{V} / \Omega)
\]

Possible solutions:
\begin{itemize}
  \item Reduce the resistor value $R$ while keeping $V_{CC}$ constant.
  \item Increase the supply voltage $V_{CC}$ while keeping $R$ constant.
  \item Increase $V_{CC}$ and reduce $R$ simultaneously.
\end{itemize}


\subsection{Role of Diode D2}

\textbf{Normal operating condition:}
\begin{itemize}
  \item RS-485 operates with a differential voltage range of approximately
  $-1.5\,\text{V}$ to $+5\,\text{V}$ (MAX485 tolerates up to $\pm5\,\text{V}$ on A/B lines).
  \item When the voltage between A and B remains within this range,
  the TVS diode behaves like an open circuit (non-conductive).
\end{itemize}

\[
\Rightarrow \text{Result: No impact on RS-485 signal integrity.}
\]

\textbf{During overvoltage events (surge/ESD/lightning):}
\begin{itemize}
  \item If line A rises more than $+3.3\,\text{V}$ relative to line B,
  the TVS diode enters breakdown mode:
  \begin{itemize}
    \item Current flows through the diode, clamping from A to B.
    \item The voltage is limited to approximately $+1.5$ to $3.3\,\text{V}$ (clamping voltage).
  \end{itemize}

  \item If line A drops below $-1.5\,\text{V}$ relative to line B,
  the diode conducts in the reverse direction and clamps the voltage.

  \item If a large surge occurs relative to GND,
  the diode conducts and diverts current to ground,
  ensuring the voltage remains within safe limits for the MAX485.
\end{itemize}

\textbf{Conclusion:}

\begin{center}
\begin{tabular}{|c|l|}
\hline
\textbf{Condition} & \textbf{Action of D2} \\
\hline
Low voltage & Non-conductive, no signal impact \\
High positive voltage & Conducts, shunts surge to ground \\
High negative voltage & Conducts in reverse, clamps voltage \\
\hline
\end{tabular}
\end{center}

\[
\Rightarrow \text{Result: RS-485 lines are protected against overvoltage and ESD.}
\]

\subsection{Using the 74HC595 IC to design a display circuit for showing values on four 7-segment LEDs}


\subsubsection*{Objective}
The circuit was designed and implemented to control four 7-segment LED displays using two 74HC595 shift registers. 
Multiplexing was applied to reduce hardware complexity and the number of I/O connections to the microcontroller.


\subsubsection*{Operating Principle}
Each 7-segment display requires eight control signals corresponding to segments \textit{a, b, c, d, e, f, g} and the decimal point (\textit{dp}). Since a single 74HC595 provides only eight output pins, it is insufficient to drive four displays simultaneously.

To solve this problem, the following architecture is used:
\begin{itemize}
    \item The \textbf{first 74HC595} controls the segment lines (\textit{a--g, dp}).
    \item The \textbf{second 74HC595} controls the digit selection signals (\textit{sel0--sel3}).
    \item The displays are driven using the \textbf{multiplexing (scanning)} method, where only one display is enabled at a time, but the switching speed is fast enough to appear continuous to the human eye.
\end{itemize}

The two 74HC595 ICs are connected in a \textbf{daisy-chain configuration}. 
The serial data output (SDO) of the first IC is connected to the serial data input (SDI) of the second IC, 
while both ICs share the same shift clock (SH\_CP) and latch clock (ST\_CP).


\subsubsection*{Circuit Connections}
\begin{itemize}
    \item The \texttt{SDI} pin of the first 74HC595 receives serial data from the microcontroller.
    \item The \texttt{SDO} pin of the first IC is connected to the \texttt{SDI} pin of the second IC.
    \item The \texttt{SH\_CP} (clock) and \texttt{ST\_CP} (latch) pins of both ICs are connected together.
    \item Outputs Q0--Q7 of the first 74HC595 drive segments \textit{a--g} and \textit{dp} through 220~$\Omega$ current-limiting resistors.
    \item Outputs of the second 74HC595 (\textit{sel0--sel3}) control NPN transistors that switch the common cathode (K) of each 7-segment display.
    \item The \texttt{OE} pin is tied to ground to enable outputs permanently, while the \texttt{CLR} pin is tied to VCC to prevent unintended reset.
\end{itemize}

\subsubsection*{Display Scanning Algorithm}
The microcontroller performs the following steps repeatedly:
\begin{enumerate}
    \item Disable all displays by clearing \textit{sel0--sel3}.
    \item Shift the segment data (\textit{a--g, dp}) for the desired digit into the first 74HC595.
    \item Shift the corresponding digit select signal into the second 74HC595.
    \item Toggle the \texttt{LATCH} signal to update all outputs simultaneously.
    \item Maintain the state for approximately 1--2~ms.
    \item Repeat the process for the remaining digits.
\end{enumerate}

A sufficiently high scanning frequency ensures stable display output without visible flickering.

\subsubsection*{Schematic and PCB Layout}

\begin{figure}[H]
  \centering
  \includegraphics[width=0.8\textwidth]{graphics/7seg_sch.png}
  \caption{Schematic of 7-segment display circuit using 74HC595}
  \label{fig:7seg_sch}
\end{figure}

\begin{figure}[H]
  \centering
  \includegraphics[width=0.8\textwidth]{graphics/7seg_pcb_2d.png}
  \caption{PCB layout of 7-segment display circuit using 74HC595}
  \label{fig:7seg_pcb_2d}
\end{figure}

\begin{figure}[H]
  \centering
  \includegraphics[width=0.8\textwidth]{graphics/7seg_pcb_3d_top.png}
  \caption{3D view of PCB layout for 7-segment display circuit using 74HC595}
  \label{fig:7seg_pcb_3d_top}
\end{figure}


\begin{figure}[H]
  \centering
  \includegraphics[width=0.8\textwidth]{graphics/7seg_pcb_3d_bottom.png}
  \caption{3D view of PCB layout for 7-segment display circuit using 74HC595}
  \label{fig:7seg_pcb_3d}
\end{figure}