\section{Introduction}
\subsection{Introduction about diode}
A diode is a two-terminal electronic component that conducts current primarily in one direction; it has low (ideally zero) resistance in one direction, and high (ideally infinite) resistance in the other. A diode vacuum tube or thermionic diode is a vacuum tube with two electrodes, a heated cathode and a plate, in which electrons can flow in only one direction, from cathode to plate. Semiconductor diodes were the first semiconductor electronic devices. The most common type of diode today is the semiconductor diode, which is a crystalline piece of semiconductor material with a p–n junction connected to two electrical terminals. Semiconductor diodes were the first semiconductor electronic devices. The most common type of diode today is the semiconductor diode, which is a crystalline piece of semiconductor material with a p–n junction connected to two electrical terminals.

Modern electronic systems mainly use semiconductor diodes. A semiconductor diode is typically made from a crystalline semiconductor material forming a p--n junction, with two electrical terminals connected to the p-type and n-type regions. Due to their small size, reliability, and efficiency, semiconductor diodes are widely used in electronic circuits today.

Diodes are widely used in electronic circuits for rectification, voltage regulation, signal demodulation, and circuit protection. For example, rectifier diodes are commonly used in power supply circuits to convert alternating current (AC) into direct current (DC).

\subsection{Introduction about BJT}

A bipolar junction transistor (BJT) is a type of transistor that uses both electron and hole charge carriers. In contrast, unipolar transistors, such as field-effect transistors (FETs), only use one type of charge carrier. BJTs are made of three layers of semiconductor material, each capable of carrying current. The three layers form two p--n junctions: the emitter-base junction and the base-collector junction.

BJTs are classified into two types based on the arrangement of the p-type and n-type materials: NPN and PNP. In an NPN transistor, a layer of p-type semiconductor (the base) is sandwiched between two n-type semiconductors (the emitter and collector). In a PNP transistor, the arrangement is reversed, with an n-type base between two p-type layers.

BJTs are widely used in electronic circuits for amplification and switching applications. They can amplify weak electrical signals, making them essential components in audio amplifiers, radio transmitters, and other communication devices. Additionally, BJTs are used as switches in digital circuits, allowing for the control of current flow in various electronic devices.

\subsection{Introduction about OPAMP}
An operational amplifier (op-amp) is a high-gain electronic voltage amplifier with a differential input and, usually, a single-ended output. In an op-amp, the output voltage is typically hundreds of thousands of times larger than the voltage difference between the input terminals. Op-amps are widely used in analog electronics for various applications, including signal conditioning, filtering, and mathematical operations such as addition, subtraction, integration, and differentiation.

Op-amps are characterized by their high input impedance and low output impedance, making them ideal for use in voltage amplification circuits. They can be configured in various ways to perform different functions, such as inverting and non-inverting amplifiers, voltage followers, summing amplifiers, and differential amplifiers.

Op-amps are commonly used in audio equipment, instrumentation, and control systems. They play a crucial role in analog signal processing, enabling the amplification and manipulation of electrical signals in a wide range of electronic devices.

