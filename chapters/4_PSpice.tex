\section{PSpice simulation}

\subsection{3.3V regulator circuit}

\textbf{Circuits function:}

The power supply circuit is designed to generate a regulated \textbf{3.3~V DC output} from a higher DC input voltage using the \textbf{TPS5430 buck converter}.  
This 3.3~V output is used to supply power for the ESP32 microcontroller and other peripheral circuits in the system.

\textbf{Simulation:}

\begin{figure}[H]
  \centering
  \includegraphics[width=0.8\textwidth]{graphics/pow_supply_sch_pspice.png}
  \caption{Power supply schematic in PSpice}
    \label{fig:pow_supply_sch_pspice}
\end{figure}

\begin{figure}[H]
  \centering
  \includegraphics[width=0.8\textwidth]{graphics/pow_supply_sim_pspice.png}
  \caption{Power supply simulation result in PSpice}
    \label{fig:pow_supply_sim_pspice}   
\end{figure}


Figure~\ref{fig:pow_supply_sim_pspice} shows the transient response of the output voltage \(V_{\text{OUT}}\).  
When the input voltage is applied, the output voltage rises smoothly from 0~V and reaches approximately \textbf{3.3~V after about 8~ms}.  
No significant overshoot or oscillation is observed during the start-up process.
The simulation results confirm that the TPS5430-based power supply operates correctly.  
The output voltage is stable at 3.3~V and the start-up response is smooth, making the circuit suitable for powering the ESP32 and related modules.

\subsection{Current sensor circuit}

\textbf{Circuits function:}

The current sensor circuit is designed to measure the AC load current indirectly using a current transformer (TA12/TA17).  
The sensed current is converted into a small AC voltage, conditioned by operational amplifiers, and shifted to a suitable DC level so that it can be safely measured by the ADC of the ESP32 microcontroller.

In this PSpice simulation, the current transformer is modeled using an AC source to represent the secondary output of TA12/TA17.  

\textbf{Simulation:}

\begin{figure}[H]
  \centering
  \includegraphics[width=0.8\textwidth]{graphics/current_sensor_sch_pspice.png}
  \caption{Current sensor schematic in PSpice}
    \label{fig:current_sensor_sch_pspice}
\end{figure}

\begin{figure}[H]
  \centering
  \includegraphics[width=0.8\textwidth]{graphics/current_sensor_sim_pspice.png}
  \caption{Current sensor simulation result in PSpice}
    \label{fig:current_sensor_sim_pspice}
\end{figure}

The green waveform represents the AC input current, while the red waveform represents the output voltage of op-amp U3A connected to the ADC input.
The output is a sinusoidal signal proportional to the input current and shifted by a DC offset to remain within the $0-3.3~V$ ADC range.

The simulation results confirm that the current sensor circuit operates correctly.
The input AC current is accurately converted into a stable, scaled voltage signal suitable for ADC measurement without signal clipping.

\subsection{LEDs circuit}


\textbf{Circuits function:}

The LED driver circuit is designed to control two indicator LEDs using transistor switches driven by digital or PWM signals from the microcontroller.  
Each LED is connected in series with a current-limiting resistor, while NPN transistors are used to safely switch the LED currents.

In this PSpice simulation, PWM voltage sources are used to emulate the control signals generated by the microcontroller GPIO pins.

\textbf{Simulation:}

\begin{figure}[H]
  \centering
  \includegraphics[width=0.8\textwidth]{graphics/led_sch_pspice.png}
  \caption{LED driver schematic in PSpice}
  \label{fig:led_sch_pspice}
\end{figure}

\begin{figure}[H]
  \centering
  \includegraphics[width=0.8\textwidth]{graphics/led_sim_pspice.png}
  \caption{LED driver simulation result in PSpice}
  \label{fig:led_sim_pspice}
\end{figure}

The green and red waveforms represent the currents flowing through the two LEDs under PWM control.
When the control signal is high, the corresponding transistor turns on and allows current to flow through the LED.
When the control signal is low, the LED current drops to nearly zero.

The simulation results show that the LED currents are properly limited to approximately 3--4~mA during the on state.
This confirms that the LED driver circuit operates correctly and is suitable for status indication applications.